\documentclass[12pt]{article}
\usepackage{amsmath}
\usepackage{textcomp}


\usepackage{gensymb}

\begin{document}

\newcommand{\mpc}{\mathrm{Mpc}}
\newcommand{\km}{\mathrm{km}}
\newcommand{\solarmass}{M_{\odot}}
\newcommand{\s}{\mathrm{sec}}
\newcommand{\is}{\sec^{-1}}
\newcommand{\ikm}{\km^{-1}}
\newcommand{\impc}{\mpc^{-1}}

\newcommand{\reh}{r_\mathrm{eh}}
\newcommand{\rh}{t_\mathrm{h}}
\newcommand{\el}{\epsilon_\Lambda}

\section*{1}

\subsection*{a}

You are getting taller, though very slowly. Consider that a tall person has a height of about two meters, or roughly \(d = 6 \times 10^{-23} \, \mpc\). The rate of height gain for this person is just \(H_0d = (70 \, \km \, \is \, \impc)(6 \times 10^{-23} \, \mpc) = 4.2 \times 10^{-21} \,\km\,\is\), which is a height gain of about one part in a trillion trillion per second.

\subsection*{b}

This is a similar idea to (a): over Earth-Sun distances, cosmological expansion is negligibly small.

\subsection*{c}

Astronomical observations appear to support a homogeneous and isotropic universe over \(100 \,\mpc\) scales. On these distance scales, there don't appear to be any special points in the universe, including our own location.

\subsection*{d}

The expansion of the universe refers to the increase in distances between any two points caused by the expansion of space itself. Even supposing there does exist matter beyond the observable horizon of every point associated with the big bang, we would not be moving toward it.

\subsection*{e}

Nowhere, or everywhere, depending on semantics. Our current position in the universe was once colocated with every other point in the universe at the time of the big bang. Again, it's important to understand that the big bang is not a classical explosion. Objects are not receding from each other because they were ejected from a central point, rather, all space is expanding.


\section{2}




\end{document}